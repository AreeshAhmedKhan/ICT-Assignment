\documentclass{article}

\usepackage{graphicx}
\usepackage{amsmath}
\usepackage{natbib}

\title{Scientific Study of the Tiger}
\author{Your Name}
\date{\today}

\begin{document}

\maketitle

\begin{abstract}
This paper provides scientific insights about the Tiger, its behavior, habits, and adaptations.
\end{abstract}

\section{Introduction}
The tiger (Panthera tigris) is one of the most powerful predators on Earth.

\section{Biology and Characteristics}
Tigers are mammals and belong to the class Mammalia.

\begin{figure}[h!]
\centering
\includegraphics[width=0.45\textwidth]{tiger1.jpg}
\caption{A Bengal tiger in the wild.}
\end{figure}

\section{Habitat}
Tigers typically live in Asia.

\section{Diet}
Tigers are carnivores that hunt deer, buffalo, and wild boar.

\begin{table}[h!]
\centering
\begin{tabular}{|c|c|}
\hline
Scientific Name & Panthera tigris \\
\hline
Class & Mammalia \\
\hline
Diet & Carnivore \\
\hline
\end{tabular}
\caption{Biological details of the tiger.}
\end{table}

\section{Hypothesis about Tiger}
We propose the following formula to estimate the tiger's lifespan:

\[
L = 5 + \frac{W}{20}
\]

Where \(L\) is lifespan and \(W\) is weight in kg.

\section{Conclusion}
Tigers are apex predators and play a key ecological role.

\bibliographystyle{plain}
\bibliography{references}

\end{document}
